\documentclass[11pt, oneside]{article}   	% use "amsart" instead of "article" for AMSLaTeX format
\usepackage{geometry}                		% See geometry.pdf to learn the layout options. There are lots.
\geometry{letterpaper}                   		% ... or a4paper or a5paper or ... 
\geometry{landscape}                		% Activate for rotated page geometry
\geometry{margin=0.3in}

%\usepackage[parfill]{parskip}    		% Activate to begin paragraphs with an empty line rather than an indent
\usepackage{graphicx}				% Use pdf, png, jpg, or eps§ with pdflatex; use eps in DVI mode
								% TeX will automatically convert eps --> pdf in pdflatex		
\usepackage{amssymb}

\usepackage{enumitem}

\usepackage{booktabs}
\usepackage{longtable}

\begin{document}

% Requires the booktabs if the memoir class is not being used
%\begin{table}[h!]
%   \centering
{
   \renewcommand{\arraystretch}{1.5}	
   \begin{longtable}{p{1.8in} p{1.8in} p{1.8in} p{1.8in} p{1.8in}}
      \toprule
      \textbf{Inputs} & \textbf{Scripts} & \textbf{Description} & \textbf{Outputs} & \textbf{Feeds into} \\ 
      \midrule
\endhead      
%
%
Transmission Lines and Capacities. \hfill \textit{OBS: no data for this now! Need to update for region, e.g. using IPM data} & \texttt{TransmissionLineFuncs} & Imports data & Transmission line source, sink, capacity & CE Model \\
%
%
Reserve Types + ??  from ??? + Michael's Storage Paper & \texttt{RIPSMaster} (\texttt{DefineReserve}...) & Sets parameters used to compute reserves, which are determined in CE + UC models & Parameters for computing reserves & CE pre-process \hfill \textit{OBS: Don't have operational reserves in CE model since there are no unit commitment constraints. Will use this for CE model.} \\
%
%
Fuel Prices from EIA AEO + EPA IPM & \texttt{RIPSMaster} & Imports Fuel Prices & Fuel Prices & CE + UCED \\
%
%
Generator Fleet from EIA 860, Needs, eGrid, PHORUM and AEO + IPM & \texttt{SetupGeneratorFleet} & Sets up generator fleet. Starts with NEEDs fleet, then adds emission rates (eGrid), cooling techs + sources (EIA 860), lat/long (eGrid), Var O\&M + Fix O\&M (AEO + IPM), unit commitment parameters (PHORUM), fuel prices, a random operational cost added (to expedite solution of optimization) + eligibility to provide regulated reserves. Script also combines plants of certain types (for computational efficiency) & Generator Fleet & CE + UCED \\
%
%
CO$_2$ cap & \texttt{InterpolateCO2Cap} & Sets CO$_2$ cap based on input limits & CO$_2$ cap & CE \\
%
%
Hourly Zonal Demand & \texttt{ForecastDemandWithRegion} & Computes hourly demand by zone given Francisco coeff + future met data from UW & Hourly Demand & Selectweeks... (DemandFuncCE) \\
%
%
New plant types for construction (see SI of storage paper for sources) & \texttt{ImportNewTechs} & Tech data compiled already in Excel from variety of sources. Script imports data, then modifies plant costs per particular cooling type with IECM data. Also filters particular plant types given inputs & Plant Types able to be built in CE & CE \\
%
%
Hourly Capacity Factors (CFs) of renewables form NREL (Wind data set + Solar Integration data set, both from NREL) \hfill \textit{OBS: should update solar to NSRDB with PVLib - Michael has downloaded NSRDB data + code for PVLib is in Solar ??? folder} 
& \texttt{GetRenewableCFs}  & 
Get hourly CFs for wind + solar by zone by:
\begin{enumerate}[leftmargin=*]
\item 'ID-ing' best wind sites
\item Getting hourly CFs for these sites
\item Replacing wind/solar in zone with best wind/solar sites until capacity zero'd out
\end{enumerate}
&
Hourly wind + solar CFs + metadata for plants to which those CFs correspond &  \texttt{GetNetDemand} $\to$ (\textit{script combines all plants + CFs into single hourly max gen profile for all wind + solar, which is input to CE + UCED}) \\
%
%
Hourly Capacity Factors (CFs) of new wind + solar that CE can build
& 
\texttt{GetRenewableCFs}  & 
Relies on \texttt{GetRenewableCFs} scripts to determine CFs for any incrementl (5 GW now) amount of capacity to existing wind + solar capacity. All wind + solar added by CE have that same avg CF profile.
&
Hourly CFs for all new wind + solar plants & CE \\
%
%
Curtailments of existing (already in fleet) thermal plant duw to ambient conditions + regulation based on UW data, Aviva coefficients + regulation limit
& \texttt{RIPSMaster} {\small\texttt{(importHourlyThermalCur..)}}  & 
For each grid cell from UW:
\begin{enumerate}[leftmargin=*]
\item Import meteo + water data
\item For each gen in cell:
\begin{enumerate}
\item set coefficients (Aviva)
\item calculate curtailment from:
\begin{enumerate}
\item ambient conditions (Aviva regression)
\item regulations (based on mixing eqn in word doc)
\end{enumerate}
\end{enumerate}
\end{enumerate}
&
Hourly time series of curtailments os each plant &  \texttt{GetHourlyCapacsForCE} \\
%
%
Hourly Zonal Demand (takes in demand coputed with Francisco data)
& 
\texttt{DemandFuncsCE} 
& 
Selects periods to include in CE model. Those inputs are:
\begin{enumerate}[leftmargin=*]
\item The day with peak net demand
\item The day with peak thermal curtailment 
\item The day with peak demand and thermal curtailment (these three itens are the "special days")
\item Factoring out those days, representative days per season. These days are selected on the basis of minimizing the RMSE between their load duration curve (LDC) and the season's LDC.
\end{enumerate}
\textit{You should explore other types of ``special days''. Also, analyze correlation between demand and curtailments.}
&
Demand (+ renewable generation) for days to be included in CE & CE \\
%
%
Weights to scale costs on representative days to full season
& 
\texttt{DemandFuncsCE} 
& 
Scalars equal to demand on representative days $\div$ ?? demand for season
&
Seasonal weights & CE \\
%
%
Max generation by each hydro plant using PNNL data
& 
\texttt{GetHydroMaxGen} 
& 
Imports monthly gen by each hydro plant from PNNL, then translates that monthly gen into max gen on each set of representative + special days for CE model. This translation is done by dividing demand on the set of days by that month's demand. The script also assigns max gen to plants NOT in the PNNL data using average capacity factor of plants in the PNNL data \hfill \textit{OBS: You should revisit this once you figure out how many days you can run the CE model for. If it is a lot of days, this approach is good. If only a few , it may not be a good idea to just scale by demand.}
&
Max gen by each hydro plant & CE \\
%
%
Curtailments of thermal plants that can be built by the CE model + regulation based on UW data, Aviva coefficients + regulation limit
& \texttt{ModifyNewTechCapacity}  & 
Similar procedure as for existing plants. Note, though, that the CE model chooses how many of each type to build in EACH CELL.
&
Hourly time series of curtailments for each tech type in each cell &  \texttt{GetHourlyCapacsForCE} \\
%
%
Mapping of cells to zones & \texttt{AssignCellsToIPMZones} 
& 
Uses shapely + shape files of IPM regions to map UW cells to IPM zones
&
Cells zone maps & CE \\
%
%
RBM data for UW
& 
\texttt{ModifyGeneratorCapac...} (\texttt{processRBMData...}) 
& 
This script processes RBM data from UW + saves water temp ($T$) data for each cell in a separate csv file. The script
\begin{enumerate}[leftmargin=*]
\item Gets a list of all cells with water temperature data from the \texttt{.spat} file
\item For each cell:
\begin{enumerate}[leftmargin=*]
\item Gets the total number of river segments in the cell + the number of days with data
\item Maps segment numbers to cell from \texttt{.spat} file
\item open the \texttt{.temp} file + reads out data for each segment in the cell
\item saves that raw data
\item Get an average temperature for each cell, which is what we use in other script, by averaging $T$ over all segments in the cell, then saves this.
\end{enumerate}
\end{enumerate}
\textit{OBS: You can use this procedure with future data from Yifan}
&
Average water temperature per segment & Scripts on curtailment \\
      \bottomrule
\end{longtable}
%   \label{tab:booktabs}
%\end{table}
}


\end{document}  