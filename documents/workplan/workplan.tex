\documentclass[11pt, oneside]{article}   	% use "amsart" instead of "article" for AMSLaTeX format
\usepackage{geometry}                		% See geometry.pdf to learn the layout options. There are lots.
\geometry{letterpaper}                   		% ... or a4paper or a5paper or ... 
%\geometry{landscape}                		% Activate for rotated page geometry
%\usepackage[parfill]{parskip}    		% Activate to begin paragraphs with an empty line rather than an indent

\usepackage{macro}
\usepackage{float}
\usepackage{booktabs}

%SetFonts

%SetFonts

\newcommand{\flow}{\text{flow}}
\newcommand{\bc}{\bar{c}}


\title{Work Plan for Capacity Expansion Model}
\author{Francisco Fonseca}
\date{December 2017}							% Activate to display a given date or no date

\begin{document}
\maketitle
\section{Introduction}

In December 2017, Michael Craig handled the remaining implementation of the RIPS Capacity Expansion (CE) model to me. This document summarizes the work plan for the next steps still needed to be implemented in the code of the model. This list is mostly based on the word document \texttt{RIPSGuide\_Craig\_7Dec17.docx} available in the git repository.

\section{To do list}

According to Michael, ``[T]he Python code for processing inputs to and outputs from the CE model is largely complete, although some debugging may be necessary.'' He goes on to list some modifications to the model that are pending:

\begin{itemize}
\item Solar data currently comes from the NREL Solar Integration Dataset. However, Bri recommends we instead use NSRDB data, which provides solar irradiance, then use that data to estimate PV generation. I have downloaded NSRDB data at points in a grid over the entire region (Databases/NSRDBRIPS). In the SolarMOEPaper folder, there are Python scripts for inputting NSRDB data to PVLib to get estimated hourly generation. (\texttt{GetRenewableCFs} script)
\item Right now, the Python code has placeholder code to insert Aviva's regressions that link capacity deratings (NOT related to regulatory limits) to ambient conditions. You will need to update the form and coefficients in these regressions and the mapping of plants to regressions.  (\texttt{CurtailmentRegressions} script)
\item The code loads meteorological data at the regional rather than cell-specific data. If you do use regressions from Aviva, you will need to use cell-specific meteorological data from UW.  (\texttt{ModifyGeneratorCapacity} script and \texttt{loadMetData} function)
\item The CE model currently uses Francisco's demand forecast for TVA. That code should be updated for the Southeast when regressions are available. \\ \texttt{ForecastDemandWithRegression}
\item The CE model has a ``specialh' set of hours, which is currently used to include hours from the day with peak demand. However, you may also want to include days with peak curtailment of generators. If so, then you will need to add a set of hours for these peak curtailment hours. I did not because the day with peak demand may very well overlap with the day with peak curtailment. (\texttt{DemandFuncsCE} script, \texttt{selectWeeksForExpansion} function (also will require modifications to GAMS code))
\end{itemize}

In addition to the list of modification that Michael pointed out, there are a few that I have been thinking:
\begin{itemize}
\item The result of this model will be the decisions of building different classes of power plants in different regions of the southeast according to demand forecast and climate related constraints. It would be interesting to implement a tool/function in order to be able to visualize these results on a plot. I have not decided what this tool would look like. But one possibility is to have a gridded map (or several gridded maps) where we could observe where the power plants are being installed. I know how to do it in R, but I need to look at how to do it in Python.
\item Paulina mentioned that the curtailment procedure should be also a function of cooling technologies. Michael's current function does not account for this.
\end{itemize}

\section{Capacity Expansion model}

\subsection{Definitions}

\begin{table}[H]
   \centering
   \caption{Decision Variables}
   \begin{tabular}{p{1in} p{4in} } % Column formatting, @{} suppresses leading/trailing space
      \toprule
      \textbf{Set} & \textbf{Definition} \\
      \midrule
      $n^{(c)}_{c, j}$ & number of new thermal generators of type $j$ in the class that CAN be curtailed (the $(c)$ superscript) built in CELL $c$\\
      $n^{(\bc)}_{z, j}$ & number of new thermal generators of type $j$ in the class that CANNOT be curtailed (the $(\bc)$ superscript) built in ZONE $z$\\
      $n^{(r)}_{z, j}$ & number of new generators of type $j$ in the class RENEWABLE (the $(r)$ superscript) built in ZONE $z$\\
      $p^{(c)}_{c, j, t}$ & electricity generation (GWh) at time $t$ of new generators  of type $j$ in the  class that CAN be curtailed (the $(c)$ superscript) built in CELL $c$\\
      $p^{(\bc)}_{z, j, t}$ & electricity generation (GWh) at time $t$ of new generators of type $j$ in the class that CANNOT be curtailed (the $(\bc)$ superscript) built in ZONE $z$\\
      $p^{(r)}_{z, j, t}$ & electricity generation (GWh) at time $t$ of new generators  of type $j$ in the class RENEWABLE (the $(r)$ superscript) built in ZONE $z$\\
      $p^{(e)}_{i, t}$ & electricity generation (GWh) at time $t$ of existing (the $(e)$ superscript) generator of index $i$ \\
      $\flow_{\ell, t}$ & flow on line $\ell$ (GW) in hour $t$\\
%
      \bottomrule
   \end{tabular}
   \label{tab:decision}
\end{table}

% Requires the booktabs if the memoir class is not being used
\begin{table}[H]
   \centering
   \caption{Sets}
   \begin{tabular}{p{1in} p{4in} } % Column formatting, @{} suppresses leading/trailing space
      \toprule
      \textbf{Set} & \textbf{Definition} \\
      \midrule
      $\Bc$ & set of user-defined time blocks. These are needed for computational purposes. $\Bc = \{\text{peak-hours, winter, summer, spring, fall, special periods}\}$ \\
      $\Ic$ & set of existing generators in the fleet. \\
      $\Ic (z)$ & subset of existing generators that are located in zone $z$.  $\Ic (z) \subseteq \Ic$\\
      $\Cc$ & set of grid cells that new techs can be placed in. \\
      $\Cc (z)$ & subset of grid cells that new techs can be placed in that are located in zone $z$. $\Cc (z) \subseteq \Cc$\\
      $\Jc$ & set of candidate plant types for new construction \\
      $\Jc^{(c)}$ & subset of plant types for new construction that can be curtailed. $\Jc^{(c)} \subseteq \Jc$\\
      $\Jc^{(\bc)}$ & subset of plant types for new construction that CANNOT be curtailed. $\Jc^{(\bc)} \subseteq \Jc$\\
      $\Jc^{(r)}$ & subset of plant types for new construction that are renewable. $\Jc^{(r)} \subseteq \Jc$\\
      $\Lc$ & set with transmission lines between load zones \\
      $\Zc$ & set with user defined load zones \\
      \bottomrule
   \end{tabular}
   \label{tab:sets}
\end{table}


% Requires the booktabs if the memoir class is not being used
\begin{table}[H]
   \centering
   \caption{Parameters}
   \begin{tabular}{p{1in} p{4in} } % Column formatting, @{} suppresses leading/trailing space
      \toprule
      \textbf{Parameter} & \textbf{Definition} \\
      \midrule
      $P^{MAX}_{c, j, t}$ & Maximum electricity generation capacity, accounting for deratings, of plant type $j \in \Jc^{(c)}$ at cell grid $c$ at time $t$ (MWh) \\
      $P^{NP}_{j}$ & Nameplate electricity generation capacity of plant type $j \in \Jc$ (MWh)\\
      $P^{MAX}_{i, t}$ & Maximum electricity generation capacity, accounting for deratings, of existing generator $i$ (non solar and non wind) at time $t$ (MWh)\\
      $P^{MAX}_{solar, t}$ & Maximum electricity generation by all existing solar generators at time $t$ (MWh) \\
      $P^{MAX}_{wind, t}$ & Maximum electricity generation by all existing wind generators at time $t$ (MWh) \\      
      \bottomrule
   \end{tabular}
   \label{tab:indices}
\end{table}


% Requires the booktabs if the memoir class is not being used
\begin{table}[H]
   \centering
   \caption{Indices}
   \begin{tabular}{p{1in} p{4in} } % Column formatting, @{} suppresses leading/trailing space
      \toprule
      \textbf{Indices} & \textbf{Definition} \\
      \midrule
      $b$ & Time blocks representing peak-hours, winter, summer, spring, fall, special periods. $b \in \Bc$\\
      $c$ & grid cells that new techs can be placed in. $c \in \Cc$ \\
      $\ell$ & Transmission Lines. $\ell \in \Lc$\\
      $i$ & existing generators in fleet. $i \in \Ic$\\
      $z$ & sub regions of SERC. $z \in \Zc$\\
      \bottomrule
   \end{tabular}
   \label{tab:indices}
\end{table}

\subsection{Objective Function}

\begin{equation} \label{eq:obj_fun}
\begin{split}
TC = &  \sum_{c \in \Cc}\sum_{j \in \Jc^{(c)}} n^{(c)}_{c, j} \times P^{NP}_{j} \times (FOC_{j} + OCC_{j} \times CRF_{j}) \\
& +  \sum_{z \in \Zc}\sum_{j \in \Jc^{(\bc)}} n^{(\bc)}_{z, j} \times P^{NP}_{j} \times (FOC_{j} + OCC_{j} \times CRF_{j}) \\
& +  \sum_{z \in \Zc}\sum_{j \in \Jc^{(r)}} n^{(r)}_{z, j} \times P^{NP}_{j} \times (FOC_{j} + OCC_{j} \times CRF_{j}) \\
& + \sum_b \bigg( W_b \sum_{t_b \in T_b}\bigg( \sum_{c \in \Cc}\sum_{j \in \Jc^{(c)}} p^{(c)}_{c, j, t_b} \times OC_{j, t_b} + \sum_{z \in \Zc}\sum_{j \in \Jc^{(\bc)}} p^{(\bc)}_{z, j, t} \times OC_{j, t_b} \\
& \qquad\qquad\qquad\qquad +  \sum_{z \in \Zc}\sum_{j \in \Jc^{(r)}} p^{(r)}_{z, j, t_b} \times OC_{j, t_b} + \sum_{i} p_{i, t_b} \times OC_{i, t_b} \bigg)\bigg)
\end{split}
\end{equation}

$CRF$ is the capital recovery ratio of each technology and is defined as:

\begin{equation} \label{eq:cap_rec}
\begin{aligned}
CRF_c = \frac{Q}{1-\left(1/(1+Q)^{D_c}\right)}
\end{aligned}
\end{equation}

\subsection{Supply vs Demand constraint}

\begin{equation} \label{eq:supply_demand}
\begin{split}
P^D_{t, z} = & \sum_{i \in \Ic(z)} p_{i, t} +  \sum_{c \in \Cc (z)}  \sum_{j \in \Jc^{(c)}} p^{(c)}_{c, j, t_b} + \sum_{j \in \Jc^{(\bc)}} p^{(\bc)}_{z, j, t_b} +  \sum_{j \in \Jc^{(r)}} p^{(r)}_{z, j, t_b} \\
& +  \sum_{\ell: \text{end}(\ell) = z} \flow_{\ell, t} - \sum_{\ell: \text{begin}(\ell) = z} \flow_{\ell, t}
\end{split}
\end{equation}

\subsection{Reserve margin constraint}

\begin{equation} \label{eq:reserve}
\begin{split}
(1+M)\times P^D_{t, z} \le & \sum_{c \in \Cc} \sum_{j \in \Jc^{(c)}} P^{MAX}_{c, j, t} \times n^{(c)}_{c, j} +   \sum_{j \in \Jc^{(r)}} P^{MAX}_{z, j, t} \times n^{(r)}_{z, j}\times CF_{j, t} \\
& + \sum_{i \in \Ic \setminus \{\Ic_w \cup \Ic_s \}} P^{MAX}_{i, t} + P^{MAX}_{\text{solar}, t} + P^{MAX}_{\text{wind}, t}
\end{split}
\end{equation}

\subsection{Maximum generation constraints}

\begin{equation} \label{eq:ex_solar}
\begin{split}
\sum_{i \in \Ic_s} p_{i, t} \le P^{MAX}_{\text{solar}, t} \quad \forall~t
\end{split}
\end{equation}

\begin{equation} \label{eq:ex_wind}
\begin{split}
\sum_{i \in \Ic_w} p_{i, t} \le P^{MAX}_{\text{wind}, t} \quad \forall~t
\end{split}
\end{equation}

\begin{equation} \label{eq:ex_therm}
\begin{split}
p_{i, t} \le P^{MAX}_{i, t} \quad \forall~t, \forall~i \in \Ic \setminus \{\Ic_w \cup \Ic_s \}
\end{split}
\end{equation}

\begin{equation} \label{eq:new_therm}
\begin{split}
p^{(c)}_{c, j, t} \le P^{MAX}_{c, j, t} \times n^{(c)}_{c, j} \quad \forall~c, t  \text{ and }\forall~j\in \Jc^{(c)}
\end{split}
\end{equation}

\begin{equation} \label{eq:new_therm2}
\begin{split}
p^{(\bc)}_{z, j, t} \le P^{MAX}_{z, j, t} \times n^{(\bc)}_{z, j} \quad \forall~z, t  \text{ and }\forall~j\in \Jc^{(\bc)}
\end{split}
\end{equation}

\begin{equation} \label{eq:new_ren}
\begin{split}
p^{(r)}_{z, j, t} \le n^{(r)}_{z, j}\times P^{NP}_{j} \times CF_{j, t}  \quad \forall~z, t  \text{ and }\forall~j\in \Jc^{(r)}
\end{split}
\end{equation}


\end{document}  