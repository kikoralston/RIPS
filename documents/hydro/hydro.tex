\documentclass[11pt, oneside]{article}   	% use "amsart" instead of "article" for AMSLaTeX format
\usepackage{geometry}                		% See geometry.pdf to learn the layout options. There are lots.
\geometry{letterpaper}                   		% ... or a4paper or a5paper or ... 
%\geometry{landscape}                		% Activate for rotated page geometry

\usepackage[parfill]{parskip}    		% Activate to begin paragraphs with an empty line rather than an indent
\usepackage{graphicx}				% Use pdf, png, jpg, or eps§ with pdflatex; use eps in DVI mode
								% TeX will automatically convert eps --> pdf in pdflatex		
\usepackage{amssymb}

\usepackage{hyperref}

\title{CE Model \\ Hydro Generation}
%\author{The Author}
\date{}							% Activate to display a given date or no date

\begin{document}
\maketitle
\section{Introduction}

This document details the methods used in the CE model to simulate generation by hydro power plants in the SERC region. It uses data generated by the MOSART model developed by PNNL.

MOSART \footnotemark is a computer tool that simulate water flows in river systems that include dam systems. In our work it is combined with the VIC and RBM models to simulate stream flow temperatures in the rivers in the SERC system taking into account the existence of reservoirs in these rivers.

\footnotetext{More information at \url{https://www.pnnl.gov/science/highlights/highlight.asp?id=1332}}

MOSART uses rule curves to estimate the water discharges in the reservoirs. When the reservoirs are also part of a hydro generation power plant we use these discharges to estimate the energy generated by the unit.

This document details the methods used for computing the energy from these hydro power plants.

\section{Calculation of monthly generation}

\begin{equation}
P_{y, m, i} = P^{base}_{m,i} \times \left( \frac{\sum_{i} q_{y,i}}{\sum_{i}q^{base}_{y, i}}\right)
\end{equation}

\begin{tabular}{l l}
$P$ &  hydro generation \\
$P^{base}$ & 2011-2015 average generation \\
$q$ & discharge \\
$q^{base}$ & base period discharge (1985-2004) \\
$y$ & year\\
$m$ & month\\
$i$ &  hydro power plant index\\
\end{tabular}

1985-2004 average discharge is chosen as baseline flow mainly due to two reasons: (1) 20-year average is more appropriate as baseline from a hydrological perspective to include both dry and wet years; and (2) 2011-2015 flow simulation uses projected climate forcing and is not validated against observation

\section{Calculation of daily generation}

\begin{itemize}
\item CMU will get daily H20 releases from Nathalie WM which we will use instead of mid term coordination (monthly to daily) to constrain daily hydropower generation.
\item Nathalie will give us daily water release (and CMU will convert it to maximum daily hydro generation). Our UC model will then control for maximum power generation using the capacity of each power plant. 
\end{itemize}

CMU will convert daily release to (maximum) daily generation by:

$$P^{MAX}_{day} = P_{month}^{PNNL}\frac{q_{day}^{PNNL}}{\sum_{day \, \in\,  month} q_{day}^{PNNL}}$$

\end{document}  

